\documentclass[11pt]{article}

%---------------------------------------------------------
%Präambel

%unterstützt Umlaute
\usepackage[utf8]{inputenc}

%unterstützt deutsche Bezeichnungen: "Abbildung" statt "Figure"
\usepackage[ngerman]{babel}

%verändert Randbreite
\usepackage{geometry}
\geometry{a4paper,left=3cm,right=3cm, top=2cm, bottom=3cm} 

%unterstützt einbinden von Grafiken
\usepackage{graphicx} 

%Grafiken mit Textumfluss
%\usepackage {picins}

%unterstützt farbliche Texte und farbliche Hintergründe
\usepackage{color}

%erlaubt weitere Farben
\usepackage[dvipsnames]{xcolor}

%leert Platzhalter für Kopft- und Fußzeilen
% setzt Linien
\usepackage[headsepline,footsepline]{scrpage2}
\pagestyle{scrheadings}
\clearscrheadfoot

%unterstützt Matrizen
\usepackage{amsmath}

%setzt Punkte im Inhaltsverzeichnis
\usepackage{tocstyle}
\newtocstyle[KOMAlike][leaders]{alldotted}{}
\usetocstyle{alldotted}

%verhindert Einrücken der Zeile nach neuem Absatz
\parindent0pt

%setzt Kopf- und Fußzeilen
\ihead{ }
\chead{\headmark}
\automark{section}
\ohead{ }
\ifoot{ }
\cfoot{\pagemark}
\ofoot{ }
%---------------------------------------------------------

\begin{document}

%Ttelseite
\begin{titlepage}
	\centering
	\includegraphics[width=0.3\textwidth]{Bilder/tub_logo.png}\par\vspace{1cm}
	{\scshape\LARGE Technische Universität Berlin \par}
	\vspace{1cm}
	{\scshape\Large Sommersemester 2018 \par}
	\vspace{1.5cm}
	{\huge\bfseries DCAITI Project\par}
	\vspace{0.5cm}
	{\Large Argumented Reality for Construction Sites and Historical Places\par}
	\vspace{2cm}
	{\Large\itshape Herbert Potechius - 353420\par}
	{\Large\itshape Max Mustermann - 000000\par}
	{\Large\itshape Max Mustermann - 000000\par}
	\vfill
	supervised by\par
	Konstantin \textsc{Klipp}
	\vfill
	% Bottom of the page
	{\large \today\par}
\end{titlepage}

%Inhaltverzeichnis
\tableofcontents

\newpage

%---------------------------------------------------------

\section{Abstract}
%TODO

\section{Use Case}
%TODO

\section{User Manual}
%TODO

\section{Technical Overview}
%TODO

\section{Planning Phase}
%TODO

%einfachen Bildeinfügen
%---------------------------------------------------------
%%%%\fbox{\includegraphics[width=0.2\textwidth]{pi.png}}
%---------------------------------------------------------


%Bild im Textfluss
%---------------------------------------------------------
%\parpic[xy] Parameter:
%	l=links, r=rechts, f=Rahmen, d=gestrichelter Rahmen, 
%	o=abgerundeter Rahmen, x=3D-Kasten, s=Schatten
%%%%\piccaption{Pikachu} 
%%%%\parpic[rd]{\includegraphics [width=4cm,height=4cm]{pi.png}}
%---------------------------------------------------------


%zentriertes Bild mit Titel
%---------------------------------------------------------
%%%%\begin{figure}[htbp] 
%%%%  \centering
%%%%     \includegraphics[width=0.3\textwidth]{pi.png}
%%%%  \caption{Erstes Bild}
%%%%  \label{fig:Bild1}
%%%%\end{figure}
%---------------------------------------------------------

\newpage

\section{Literatur}

\begin{itemize}
	\item https://www.youtube.com/watch?v=3kemRzfilzk
	\item https://de.wikibooks.org/wiki/LaTeX-W
\end{itemize}

\end{document}